\documentclass[conference]{IEEEtran}
\usepackage[utf8]{inputenc}
\usepackage{glossaries}
\usepackage{flushend}

\ifCLASSINFOpdf
  \usepackage[pdftex]{graphicx}
  \graphicspath{.}
\else
\fi

\newacronym{wave}{WAVE}{Waveform Audio File Format}

\begin{document}
\title{Heartbeat Sound Classification}

\author{\IEEEauthorblockN{Gustavo Gomes}
\IEEEauthorblockA{Faculty of Engineering\\
University of Beira Interior\\
Email: a30232@ubi.pt}
\and
\IEEEauthorblockN{José Monteiro}
\IEEEauthorblockA{Faculty of Engineering\\
University of Beira Interior\\
Email: a31366@ubi.pt}
\and
\IEEEauthorblockN{Pedro Silvestre}
\IEEEauthorblockA{Faculty of Engineering\\
University of Beira Interior\\
Email: a31269@ubi.pt}
}

\maketitle

\begin{abstract}
Heart problems are one of the major causes of death world wide, therefore it's
crucial to detect such problems efficiently. The answer is to automate the
detection of anomalies in order to correctly detect individuals who are at risk
and consequently prevent further deaths. To that extent we were asked to develop
a machine learning system, that can properly classify sound files, containing
heartbeats and identify if they seem to come from a subject with a heart defect
or not.
\end{abstract}

\IEEEpeerreviewmaketitle

\section{Introduction}
In the beginning of the current semester we were proposed to develop a system
that can detect heart defects in sound files. To that end we received
\acrfull{wave} files containing the sound of heartbeats and process them in
order to detect every heartbeat in them. Afterwards, we should analyze the
patterns of those heartbeats and try to classify them according to a test
dataset that we had also received.

\section{Implementation Decisions}
To achieve the expected results proposed by the professor we chose to use Python
and also after some research we stumbled upon some python libraries and we made
use of the following packages:
\begin{itemize}
	\item \textit{Numpy}
	\item \textit{Scipy}
	\item \textit{Matplotlib}
	\item \textit{CSV}
	\item \textit{scikit-learn}
\end{itemize}

We have firstly started using Python's lists and some coded functions that would
do all the calculations. When we received the second dataset for testing, the
need for more efficiency that arose because the lists started to become a tad
slow. That was when we decided to implement our functions in \textit{Numpy}
arrays and use the syntactical sugar that came with it. We have decided to
skip the Shannon's energy, filtering step because it would stop our system from
distinguishing the two different types of heartbeats. So to that extent we
ended up only performing the following steps of the recommended filtering
process:
\begin{enumerate}
	\item Frame Rate conversion
	\item Moving average
	\item Normalization
\end{enumerate}

\section{Critical Review}
Our project is becoming more and more complex and we had a few difficulties
because we didn't prepare our data in a way that was maintainable so with the
change of our data, from heartbeats to the four parameters that were asked for
the classification problems we ended up making a lot of changes to our code.

We would have wanted to make a genetic algorithm and run our neural network with
the goal of find any type of correlations between the parameters we were testing
and the different classes. This would be helpful to find if one or more of our
parameters are more relevant to a specific class, and this would be some
valuable insights towards heart deffect detection.

Because we are very critical about our work we thinkk that our work isn't as
good as it could be since we still don't have a perfect algorithm for pulse
verification and we also didn't had too much attention to the optimization of
our classifiers so it is a system that actually takes some time to process the
data, even for just one file which is quite bad in our standards.

\begin{table}
\label{table:Error results}
\begin{tabular}{l | r }
	Filename & Error \\
	\hline
	150\_{}1306776340746\_{}B.csv & 2695.5 \\
	109\_{}1305653646620\_{}C.csv & 681.5 \\
	154\_{}1306935608852\_{}B.csv & 573.0 \\
	150\_{}1306776340746\_{}C.csv & 621.0 \\
	139\_{}1306519274653\_{}A.csv & 34.0 \\
	143\_{}1306763822290\_{}C.csv & 10.0 \\
	128\_{}1306344005749\_{}B.csv & 528.0 \\
	128\_{}1306344005749\_{}D1.csv & 23.0 \\
	153\_{}1306848820671\_{}B1.csv & 514.0 \\
	128\_{}1306344005749\_{}D.csv & 8.0 \\
	159\_{}1307018640315\_{}A.csv & 12.5 \\
	115\_{}1306259437619\_{}C.csv & 628.0 \\
	133\_{}1306759619127\_{}D.csv & 615.5 \\
	146\_{}1306778707532\_{}D4.csv & 1489.0 \\
	146\_{}1306778707532\_{}D2.csv & 524.5 \\
	126\_{}1306777102824\_{}D.csv & 632.5 \\
	129\_{}1306344506305\_{}D1.csv & 589.5 \\
	137\_{}1306764999211\_{}D.csv & 355.0 \\
	141\_{}1306520154450\_{}B.csv & 451.5 \\
	106\_{}1306776721273\_{}C1.csv & 649.5 \\
	133\_{}1306759619127\_{}B.csv & 11.5 \\
	154\_{}1306935608852\_{}D.csv & 1738.0 \\
	134\_{}1306428161797\_{}C.csv & 3279.5 \\
	152\_{}1306779561195\_{}D.csv & 36.5 \\
	154\_{}1306935608852\_{}A1.csv & 4.0 \\
	113\_{}1306244002866\_{}D.csv & 37.5 \\
	168\_{}1307970069434\_{}A2.csv & 45.5 \\
	146\_{}1306778707532\_{}A.csv & 521.5 \\
	153\_{}1306848820671\_{}B.csv & 2633.5 \\
	103\_{}1305031931979\_{}D3.csv & 4591.5 \\
	140\_{}1306519735121\_{}D1.csv & 28.0 \\
	129\_{}1306344506305\_{}B1.csv & 44.5 \\
	145\_{}1307987561278\_{}B.csv & 654.0 \\
	106\_{}1306776721273\_{}D1.csv & 674.5 \\
	146\_{}1306778707532\_{}D1.csv & 24607.5 \\
	149\_{}1306776016110\_{}B.csv & 37.0 \\
	151\_{}1306779785624\_{}A.csv & 524.5 \\
	159\_{}1307018640315\_{}C1.csv & 24.0 \\
	152\_{}1306779561195\_{}C1.csv & 16.5 \\
	152\_{}1306779561195\_{}B1.csv & 840.5 \\
	153\_{}1306848820671\_{}A.csv & 549.5 \\
	134\_{}1306428161797\_{}D.csv & 32.0 \\
	127\_{}1306764300147\_{}B.csv & 1859.0 \\
	145\_{}1307987561278\_{}C.csv & 20.0 \\
	125\_{}1306332456645\_{}B.csv & 316.5 \\
	158\_{}1306947254705\_{}B2.csv & 579.5 \\
	168\_{}1307970069434\_{}A.csv & 608.5 \\
	153\_{}1306848820671\_{}D1.csv & 1631.0 \\
	147\_{}1306523973811\_{}C.csv & 575.5 \\
	149\_{}1306776016110\_{}C1.csv & 4476.5 \\
	148\_{}1306768801551\_{}C1.csv & 524.5 \\
	141\_{}1306520154450\_{}C.csv & 482.0 \\
	155\_{}1306935902813\_{}B1.csv & 9.0 \\
	117\_{}1306262456650\_{}B.csv & 1819.0 \\
	140\_{}1306519735121\_{}A.csv & 567.5 \\
	143\_{}1306763822290\_{}B.csv & 27.0 \\
	103\_{}1305031931979\_{}D1.csv & 57.5 \\
	127\_{}1306764300147\_{}A.csv & 735.5 \\
	127\_{}1306764300147\_{}C1.csv & 308.5 \\
	129\_{}1306344506305\_{}D.csv & 97.0 \\

\end{tabular}
\end{table}

\section{Conclusion}
This project as widened our view of what Artificial Intelligence meant to us,
making us more aware of the vast possibilities that can be done using the
classifiers for example and also the neural netwrok. Before it we thought that
Artificial Intelligence was just like in the movies where a machine can think
and learn like a human being, but now we have started to get very interested in
some areas like for example Natural Language Processing, or Machine Learning and
we have also started to look at deep learning because of some media attention
it is getting these days.

We also agreed that this was an intersting topic because we never knew that we
could apply concept of Artificial Intelligence towards health and after this
course we realized that there could be a great number of applications towards
medical research and disease detection.

\section*{Acknowledgment}
The authors would like to thank Professor Paulo A. P. Fazendeiro for the
opportunity to work with real world problems for this project.

We would also like to thank Guido von Rossum for the creation of Python, because
without it the authors feel that the amount of hours they would require to make
such project.

\end{document}
