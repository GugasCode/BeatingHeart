\documentclass[conference]{IEEEtran}
\usepackage[utf8]{inputenc}
\usepackage{glossaries}
\usepackage{flushend}

\ifCLASSINFOpdf
  \usepackage[pdftex]{graphicx}
  \graphicspath{.}
\else
\fi

\newacronym{wave}{WAVE}{Waveform Audio File Format}

\begin{document}
\title{Heartbeat Sound Classification}

\author{\IEEEauthorblockN{Gustavo Gomes}
\IEEEauthorblockA{Faculty of Engineering\\
University of Beira Interior\\
Email: a30232@ubi.pt}
\and
\IEEEauthorblockN{José Monteiro}
\IEEEauthorblockA{Faculty of Engineering\\
University of Beira Interior\\
Email: a31366@ubi.pt}
\and
\IEEEauthorblockN{Pedro Silvestre}
\IEEEauthorblockA{Faculty of Engineering\\
University of Beira Interior\\
Email: a31269@ubi.pt}
}

\maketitle

\begin{abstract}
Heart problems are one of the major causes of death world wide, therefore it's
crucial to detect such problems efficiently. The answer is to automate the
detection of anomalies in order to correctly detect individuals who are at risk
and consequently prevent further deaths. To that extent we were asked to develop
a machine learning system, that can properly classify sound files, containing
heartbeats and identify if they seem to come from a subject with a heart defect
or not.
\end{abstract}

\IEEEpeerreviewmaketitle

\section{Introduction}
In the beginning of the current semester we were proposed to develop a system
that can detect heart defects in sound files. To that end we received
\acrfull{wave} files containing the sound of heartbeats and process them in
order to detect every heartbeat in them. Afterwards, we should analyze the
patterns of those heartbeats and try to classify them according to a test
dataset that we had also received.

\section{Implementation Decisions}
To achieve the expected results proposed by the professor we chose to use Python
and also after some research we stumbled upon some python libraries and we made
use of the following packages:
\begin{itemize}
	\item \textit{Numpy}
	\item \textit{Scipy}
	\item \textit{Matplotlib}
	\item \textit{CSV}
	\item \textit{scikit-learn}
\end{itemize}


% not sure where to put this
Our project is made out of various modules:
\begin{itemize}
	\item \textit{ml}
	\item \textit{classifiers}
	\item \textit{charts}
	\item \textit{files}
	\item \textit{filters}
	\item \textit{waves}
	\item \textit{pulses}
\end{itemize}
% "this" ends here

We have firstly started using Python's lists and some coded functions that would
do all the calculations. When we received the second dataset for testing, the
need for more efficiency that arose because the lists started to become a tad
slow. That was when we decided to implement our functions in \textit{Numpy}
arrays and use the syntactical sugar that came with it. We have decided to
skip the Shannon's energy, filtering step because it would stop our system from
distinguishing the two different types of heartbeats. So to that extent we
ended up only performing the following steps of the recommended filtering
process:
\begin{enumerate}
	\item Frame Rate conversion
	\item Moving average
	\item Normalization
\end{enumerate}

% New stuff
For the next part of our project, the classification problem we have decided to
create one class for the two classifications that were asked, k-Nearest-Neighbors
and Naive Bayes. With this decision we wanted to make our job much easier
because we only need to create the object and pad it our training set and
afterwards simply pass every element of our test set and it will abstract the
programmer from worrying about the underlying work of the classification.

We have also decided to put the data into a structure that is basically composed
of the name of the file, all of the parameters and at the end of every data
item is the classification that the file has. We also made the decision to
simply choose the first pair of every file, we also thought about making the
mean of every pair of S1 and S2 because it would be more uniform, but made the
decision to only use the first in every file.

One thing that might be different from the other classifiers is that we have
functions to format the data, and to divide each file into tuples that account
for the four parameters.

We have also decided to make a class for our Neural Network, because we want to
identify which attributes of the problem at hand. Unfortunately at the time of
writing our classifiers are not prepared to make weighted classification so we
wore unable to test this feature and we can't publish the results.
Basically we will give each one of the four parameters a weight and our
classifiers will use it to infer new conclusions over the data and afterwards
we check for the error of the classification steps.

\section{Critical Review}
Our project is becoming more and more complex and we had a few difficulties
because we didn't prepare our data in a way that was maintainable so with the
eventual change in our data, from heartbeats to the four parameters that were
asked for the classification problems we ended up making a lot of changes to our
code.

We would have wanted to make a genetic algorithm and run our neural network with
the goal of find any type of correlations between the parameters we were testing
and the different classes. This would be helpful to find if one or more of our
parameters are more relevant to a specific class, and this would be some
valuable insights towards heart defect detection.

Because we are very critical about our work we think that our work isn't as
good as it could be since we still don't have a perfect algorithm for pulse
verification and we also didn't had too much attention to the optimization of
our classifiers so it is a system that actually takes some time to process the
data, even for just one test file which is quite bad in our standards.

The project itself is interesting but we must say that we could have made much
more of an effort towards making something that could be a production grade
application with multiple algorithms and a bigger abstraction of the input data
to aid us in future projects.

\section{Conclusion}
This project as widened our view of what Artificial Intelligence meant to us,
making us more aware of the vast possibilities that can be done using the
classifiers for example and also the neural network. Before it we thought that
Artificial Intelligence was just like in the movies where a machine can think
and learn like a human being, but now we have started to get very interested in
some areas like for example Natural Language Processing, or Machine Learning and
we have also started to look at deep learning because of some media attention
it is getting these days.

We also agreed that this was an interesting topic because we never knew that we
could apply concept of Artificial Intelligence towards health and after this
course we realized that there could be a great number of applications towards
medical research and disease detection.

\section*{Acknowledgment}
The authors would like to thank Professor Paulo A. P. Fazendeiro for the
opportunity to work with real world problems for this project.

We would also like to thank Guido von Rossum for the creation of Python, because
without it the authors feel that the amount of hours they would require to make
such project.

\end{document}
